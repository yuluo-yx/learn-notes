% 导言区 引入文档类
\documentclass[12]{article} %book report, letter
% 引入中文的宏包
\usepackage{ctex}
% 文章标题
\title{latex基础练习文档}
% 作者
% 指定字体
\author{\kaishu 姬世文}
% 日期
\date{\today}

%新建字体设置命令
\newcommand{\myfont}{\textit{\textbf{\textsf{Fancy Text}}}}

%正文区 (文稿区)
\begin{document}
	\maketitle

	hello world!

	Let $f(x)$ be defined by the formula
	$$f(x)=3x^2+x-1$$ which is a polynomial of degree 2.

	% equation环境
	\begin{equation}
	AB^2 = BC^2 + AC^2
	\end{equation}

	% 字体族的设置(罗马字体 无衬线字体 打字机字体)

	% 设置为罗马字体
	\textrm{Roman Family}

	\textsf{Sans Serif Family}

	\texttt{Typewriter Family}

	% 通过声明,声明后续的字体为罗马字体
	\rmfamily Roamn Family

	% 通过大括号的限定来声明作用字体的范围
	{\sffamily who you are? you find self on everyone around!}
	
	{\ttfamily who you are? you find self on everyone around!}
	
	% 字体系列设置(粗细,宽度)
	\textmd{Medium Series} \textbf{Boldface Series}

	{\mdseries Medium Series} {\bfseries Boldface Series}

	% 字体形状设置(直立,斜体,伪斜体,小型大写)

	% 字体设置命令	
	\textup{Upright Shape} \textit{Italic Shape}
	\textsl{Slanted Shape} \textsc{Small Caps Shape}

	% 字体设置声明
	{\upshape Upright Shape} {\itshape Italic Shape} 
	{\slshape Slanted Shape} {\scshape Small Caps Shape}
	
	% 中文字体设置
	{\songti 宋体} \quad
	{\heiti 黑体} \quad
	{\fangsong 仿宋} \quad
	{\kaishu 楷书} \quad

	% 中文字体的粗体与斜体设置
	中文字体的\textbf{粗体}与\textit{斜体}

	% 字体大小设置
	{\tiny 				yuluo}\\
	{\scriptsize 		yuluo}\\
	{\footnotesize  yuluo}\\
	{\small				yuluo}\\
	{\normalsize    yuluo}\\
	{\large 			yuluo}\\
	{\Large 			yuluo}\\
	{\huge 				yuluo}\\
	{\Huge				yuluo}\\

	% 中文字号设置命令
	\zihao{-0} 你好!

	\myfont

\end{document}