% 导言区
\documentclass{ctexart}

\usepackage{ctex}
\usepackage{amsmath}

% 正文区(文稿区)
\begin{document}

	\section{简介}
	LaTex将排版内容分为文本模式和数学模式。文本模式用于普通文本的排版,数学模式用于数学公式的排版
	\section{行内公式}
	\subsection{美元符号}
		交换律是$a+b=b+a$,如\(1+2=2+1=3\)
	\subsection{小括号}
		交换律是\(a+b=b+a\),如\(1+2=2+1=3\)
	\subsection{math环境}
		\begin{math}
			a+b=b+a
		\end{math}
		,如
		\begin{math}
			1+2=2+1=3
		\end{math}
	\section{上下标}
	\subsection{上标}
		$3x^2 - x + 2 = 0$
		% 如果x的指数是两位数,要用大括号进行分组,确保正确的格式
		$3x^{20} - x + 2 = 0$
		% 在LaTex中也可以将一个已有的公式进行上标处理,要用大括号
		$3x^{3x^2 - x + 2} - x + 2 = 0$
	\subsection{下标}
		a_0, a_1, a_2
		% 同样有多个数字时,要使用大括号
		a_{10}, a_{100}, a_{3x^2 - x + 2}
	\section{希腊字母}
		$\alpha$
		$\beta$
		$\gamma$
		$\epsilon$
		$\pi$
		$\omega$

		$\Gamma$
		$\Delete$
		$\Theta$
		$\Pi$
		$\Omega$

		% 也可以用在数学公式的排版中
		$\alpha^3 + \beta^2 + \gamma = 0$

	\section{数学函数}
		$\log$
		$\sin$
		$\cos$
		$\arcsin$
		$\arccon$
		$\ln$

		$\sin^2 x + \cos^2 x = 1$

		y = \arcsin x$
	
		$y = \ln x$

		$y = \long_2 x$

		$\sqrt{2}$
		$\sqrt{x^2 +y^2}$
		$\sqrt{x + \sqrt{2}}$
		% 4次根号下x
		$\sqrt[4]{x}$

	\section{分式}

	大约是原体积的$3/4$
	% 前一个是分子,后一个是分母
	大约是原体积的$\frac{3}{4}$
	
	\section{行间公式}
	\subsection{美元符号}

	$$a+b=b+a$$
	如
	$$1+2=2+1=3$$

	\subsection{中括号}

	交换律是\[a+b=b+a\]
	如
	\[1+2=2+1=3\]

	\subsection{displaymath环境}

	\begin{displaymath}
	a+b=b+a
	\end{displaymath}
	如
	\begin{displaymath}
	1+2=2+1=3
	\end{displaymath}

	\subsection{自动编号公式equation环境}
	交换律见公式\ref{eq:commutative}
	% 为使公式在文本中自动编号需要使用equation环境
	\begin{equation}
	% 也可以对equation中的公式用\label命令添加标签
	a+b=b+a \label{eq:commutative}
	\end{equation}

	% 注意:带*的equation环境需要使用amsmath宏包
	\subsection{不编号公式equation*环境}

	交换律见公式\ref{eq:commutative2}

	% 如果不需要对公式进行编号,可以使用带*号的equation环境
	% 没有编号的公式在交叉引用的时候使用的是小节号
	\begin{equation*}
	a+b=b+a \label{eq:commutative2}
	\end{equation*}

\end{document}