% 导言区
\documentclass{ctexbook}

\usepackage{ctex}
%=========设置标题的格式=========
\ctexset{
	section = {
		format+ = \zihao{-4} \heiti \raggedright,
		name = {,、},
		number = \chinese{section},
		beforeskip = 1.0ex plus 0.2ex minus .2ex,
		afterskip = 1.0ex plus 0.2ex minus .2ex,
		aftername = \hspace{0pt}
		},
		subsection = {
			format+ = \zihao{-4} \heiti \raggedright,
			% name = {\thesubsection、},
			name = {,、},
			number = \arabic{subsection},
			beforeskip = 1.0ex plus 0.2ex minus .2ex,
			afterskip = 1.0ex plus 0.2ex minus .2ex,
			aftername = \hspace{0pt}
		}
}

% 正文区(文稿区)
\begin{document}

	\tableofcontents
	% 构建章节
	\chapter{绪论}
	% 构建小节
	\section{研究的目的和意义}
	\section{国内外研究现状}
	\subsection{国外研究现状}
	\subsection{国内研究现状}
	\section{研究内容}
	\section{研究方法与技术路线}
	\subsection{研究方法}
	\subsection{技术路线}

	\chapter{实验与结果分析}
	\section{引言}
	\subsection{数据}
	\subsection{图表}
	% 利用subsection构建子小节
	\subsubsection{实验条件}
	\subsubsection{实验过程}
	\subsection{结果分析}
	\section{结论}
	\section{致谢}



\end{document}