% 导言区
\documentclass{ctexart}
\usepackage{ctex}

% 正文区
\begin{document}

% 一次管理每一次使用
% 参考文献格式:
% \begin{thebibliography}{编号样本}
% 		\bibitem[记号]{引用标志}文献条目1
% 		\bibitem[记号]{引用标志}文献条目2
% 		\bibitem[记号]{引用标志}文献条目3
%		……
% \end{thebibliography}
% 	其中文献条目包括:作者,题目,出版社,年代,版本,页码等
% 引用时候要可以采用:\cite{引用标志1,引用标志2}

\begin{thebibliography}{99}
	\bibitem{atricle1}陈立辉,苏伟,蔡川,陈小云.\emph{基于LaTex的Web数学公式提取方法研究}[J].计算机科学.2014(06)
	\bibitem{book}William H. Press,Saul A. Teukolsky,
	William T. Vetterling, Brian P. Flannery,
	\emph{Numberical Recipes 3rd Edition:
	The Art of Scientific Computing}
	Cambridge University Press, New York,2007.
	\bibitem{latexGuide} Kopka Helmut, W. Daly Patrick,
	\emph{Guide to \LaTex}, $4^{th}$ Edition.
	Available at \texttt{http://www.amazong.com}.
	\bibitem{latexMath} Graetzer George, \emph{Math Into \LaTex},
	BirkhAuser Boston; 3 edition ()June 22, 2000).
\end{thebibliography}

% 需要在tex stdio文件进行相应设置
% biblatex/biber
% 新的TEX惨开文献排版引擎
% 样式文件(参考文献样式文件 --bbx文件,引用样式文件--cbx文件)使用LATEX编写
% 支持根据本地化排版,如:
%		biber -l zh_pinyin texfile, 用于指定按拼音排序
% 		biber -l zh_stroke texfile, 用于按笔画排序

\end{document}