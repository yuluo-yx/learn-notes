% 导言区
\documentclass{article}

\usepackage{ctex}

% 正文区(文稿区)
\begin{document}
	\section{latex中的特殊符号}
	\subsection{空白符号}
	% 1em(当前字体中M的宽度)
	1a\quad b

	% 2em
	2a\qquad b

	% 约为1/6个em
	3a\,b a\thinspace b

	% 0.5个em
	4a\enspace b

	% 输出一个空格
	5a\ b

	% 硬空格(不能分割的空格)
	6a~b
	
	% 指定空格长度
	7a \kern 1pc b

	8a \kern -1em b

	9a \hskip 1em b

	10a \hspace{35pt}b
	
	% 占位宽度空白
	11a\hphantom{xyz}b

	% 弹性长度空白
	12a\hfill b

	\subsection{LaTex控制符}
	
	\# \$ \% \{ \} \~{} \_{} \^{} \textbackslash \&	

	\subsection{排版符号}
	
	\S \P \dag \ddag \copyright \pounds

	\subsection{Tex标志符号}

	\Tex{} \LaTex{} \LaTeXe{}

	\subsection{引号}

	% 1 左边的`(撇号)
	` 
	% 单引号字符表示右单引号
	'	
	% 连续两个撇号表示左双引号
	``
	% 连续两个单引号表示右双引号
	''
	``你好''

	\subsection{连字符}

	% 短连字符
	-
	% 中连字符
	--
	% 长连字符
	---

	\subsection{非英文字符}

	\oe \OE \ae \AE \aa \AA \o \O \l \L \ss \SS !` ?`

	\subsection{重音符号(以o为例)}
	
	\`o \'o \^o \''o \~o \=o \.o \u{o} \v{o} \h{o} \r{o} \t{o} \b{o} \c{o} \d{o}

\end{document}