% 导言区
\documentclass{ctexart}

\usepackage{ctex}
\usepackage{amsmath}

% 定义新命令
%\newcommand{\adots}{\mathinner}{\mkern2mu%
%					\raisebox{0.1em}{.}\mkern2mu\raisebox{0.4em{.}%
%					\mkern2mu\raisebox{0.7em{.}\mkern1mu}
%					}

% 正文区(文稿区)
\begin{document}

	% 在LaTex中可以使用matrix环境实现矩阵的排版,需要引入amsmath宏包,与表格环境类似,也可以使用&分割列,\\ 分割行
	\[
		\begin{matrix}
			0 & 1 \\
			1 & 0
		\end{matrix}
	\]

	% pmatrix 用于在矩阵两边加小括号
		\[
		\begin{pmatrix}
			0 & 1 \\
			1 & 0
		\end{pmatrix}
	\]

	% bmatrix 用于在矩阵两边加中括号
		\[
		\begin{bmatrix}
			0 & 1 \\
			1 & 0
		\end{bmatrix}
	\]

	% Bmatrix 用于在矩阵两边加大括号
		\[
		\begin{Bmatrix}
			0 & 1 \\
			1 & 0
		\end{Bmatrix}
	\]	

	% vatrix 用于在矩阵两边加单竖线
		\[
		\begin{vmatrix}
			0 & 1 \\
			1 & 0
		\end{vmatrix}
	\]

	% Vatrix 用于在矩阵两边加双竖线
		\[
		\begin{Vmatrix}
			0 & 1 \\
			1 & 0
		\end{Vmatrix}
	\]

	% 使用上下标
			\[
		\begin{pmatrix}
			a_{11}^2 & a_{12}^2 & a_{13}^2 \\
			0 & a_{22} & a_{23}\\
			0 & 0 & a_{33}
		\end{pmatrix}
	\]
	
	% 矩阵中常用的省略号可以使用\dots, \vdots, \ddots
	% 在数学模式中可以使用\time命令排版乘号
	% 使用\adots实现从右到左的点号,需要自定义一个新命令
	\[
		\begin{bmatrix}
			a_{11} & \dots & a_{1n}\\
%			\adots & \ddots & \vdots \\
			& \ddots & \vdots \\
			0 & & a_{nn}
		\end{bmatrix}_{n \times n}
	\]

	% 分块矩阵,嵌套矩阵
	% \text命令用于在数学模式中临时切换到文本模式
	\[
		\begin{pmatrix}
			\begin{matrix} 1 & 0 \\ 0 & 1 \end{matrix}
			& \text{\Large 0} \\
			\text{\Large 0} & \begin{matrix}
			1 & 0 \\ 0 & -1 \end{matrix}
		\end{pmatrix}
	\]

	% 三角矩阵
	% 可以使用\multicolumn合并多列 \raisebox合并多列
	\[\begin{pmatrix}
	a_{11} & a_{12} & \cdots & a_{1n} \\
	& a_{22} & \cdots & a_{2n} \\
	&        & \ddots & \vdots \\
	\multicolumn{2}{c}{\raisebox{1.3ex}[0pt]{\Huge 0}}
	&        & a_{nn}
	\end{pmatrix}
	\]

	% 跨列的省略号:\hdotsfor{<列数>}
	\[
	\begin{pmatrix}	
	1 & \frac 12 & \dots & \frac 1n \\
	\hdotsfor{4} \\
	m & \frac m2 & \dots & \frac mn
	\end{pmatrix}
	\]
	
	% 行内小矩阵 (smallmatrix)环境
	复数 & z = (x,y)& 也可以用矩阵
	\begin{math}
	\left(   % 需要手动加上左括号
	\begin{smallmatrix}
	x & -y \\ y & x
	\end{smallmatrix}
	\right) % 需要手动加上右括号
	\end{math} 来表示

	% array环境(类似于表格环境tabular)
	% 如果需要多个字母构成分母,需要{}分组
	\[
		\begin{array}{r|r}
		\frac12 & 0 \\
		\hline
		0 & -\frac abc \\
		\end{array}
	\]
	
	% 用array环境构造复杂环境
	\[
	% @{<内容>} -添加任意内容,不占表项计数
	% 此处添加一个负值空白,表示向左移-5pt的距离
	\begin{array}{c@{\hspace{-5pt}}l]
	% 第一行,第一列
	\left(
	\begin{array}{ccc|ccc}
	a & \cdots & a & b & \cdots & b \\
	& \ddots & \vdots & \vdots & \adots \\
	&				& a & b\\ \hline
	&				&   & c & \cdots & c\\
	&				&   & \vdots & & \vdots \\
	\multicolumn{3}{c|}{\raisebox}{2ex}[0pt]{\Huge 0}}
	& c & \cdots & c
	\end{array}
	\right
	& 
	% 第一行第二列
	\begin{array}{1}
	% \left.仅表示与\right\ 配对,什么都不输出
	\left.\rule{0mm}{7mm}\right\}p \\
	\\
	\left.\rule{0mm}{7mm}\right\}q
	\end{array}
	\\[-5pt]
	% 第二行第一列
	% \underbrace命令 排版横向大括号
	\begin{array}{cc}
	\underbrace{\rule{17mm}{0mm}}_m &
	\underbrace{\rule{17mm}{0mm}}_m 
	\end{array}
	% 第二行第二列
	\end{array}
	\]

\end{document}